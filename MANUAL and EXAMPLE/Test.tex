\usemodule[basicexam]%[mode=teacher]
%\usemodule[basicexam][mode={teacher,check}]
\starttext
\section{Choice}

\setupquestion[pointlabel={,points},answerprelabel=answer:,]
\setupproblem[left=(,right=),stopper=,width=\widthofstring{(00)}]
\setupanswer[label={Answer},pointlabel={points},
             scriptcontent={\centerline{\lightgray\ss WRITE YOUR ANSWER HERE.}\par\null\vskip2\baselineskip\relax}]

\startbuffer[L-1]
\startcitem Lorem ipsum\stopcitem
\stopbuffer
\startbuffer[L-2]
\startcitem dolor sit amet\stopcitem
\stopbuffer
\startbuffer[L-3]
\startcitem[*] Lorem ipsum\stopcitem
\stopbuffer
\startbuffer[L-4]
\startcitem dolor sit amet\stopcitem
\stopbuffer

\startbuffer[AL-1]
\startcitem Lorem ipsum\stopcitem
\stopbuffer
\startbuffer[AL-2]
\startcitem dolor sit amet\stopcitem
\stopbuffer
\startbuffer[AL-3]
\startcitem[*] Lorem ipsum\stopcitem
\stopbuffer
\startbuffer[AL-4]
\startcitem dolor sit amet\stopcitem
\stopbuffer

\startbuffer[answer]
\unprotect
\ifmode_showanswer
  \setupanswer[showanswer=true]
\else
  \setupanswer[showscript=true,showanswer=false]
\fi
\protect
\startanswer
 Here is answer.
\stopanswer
\stopbuffer

\setupquestion[showanswer=false,point=2,showpoint=false]

\startquestion
  Lorem ipsum dolor sit amet,\fillin{K}
  \startchoice
    \getbuffer[L-1]\getbuffer[L-2]\getbuffer[L-3]\getbuffer[L-4]
  \stopchoice
  \getbuffer[answer]
\stopquestion

\startquestion
  Lorem ipsum dolor sit amet,
  \startchoice
    \getbuffer[L-3]\getbuffer[L-2]\getbuffer[L-3]\getbuffer[L-4]
  \stopchoice
  \getbuffer[answer]
\stopquestion

\startquestion
  Lorem ipsum dolor sit amet,
  \fastchoice{Lorem ipsum,{[*]Lorem ipsum},Lorem ipsum,Lorem ipsum}
  \getbuffer[answer]
\stopquestion

\startquestion
  Lorem ipsum dolor sit amet,
  \fastchoice{Lorem ipsum,{[*]Lorem ipsum},Lorem ipsum,Lorem ipsum}
  \getbuffer[answer]
\stopquestion

\startquestion
  Lorem ipsum dolor sit amet,
  \startinlinechoice
    \getbuffer[AL-1]\getbuffer[AL-2]\getbuffer[AL-3]\getbuffer[AL-4]
  \stopinlinechoice
  \getbuffer[answer]
\stopquestion

\startquestion[showanswer=false,point=2,showpoint=false]
  Lorem ipsum dolor sit amet,
  \fastchoicealt{Lorem ipsum,{[*]Lorem ipsum},Lorem ipsum,Lorem ipsum}
  \getbuffer[answer]
\stopquestion

\startquestion
  Lorem ipsum dolor sit amet,
  \fastchoicealt{Lorem ipsum,{[*]Lorem ipsum},Lorem ipsum,Lorem ipsum}
  \getbuffer[answer]
\stopquestion

\setupquestion[showanswer=true,point=2,showpoint=true]

\startquestion
  Lorem ipsum dolor sit amet,\fillin{K}
  \startchoice
    \getbuffer[L-1]\getbuffer[L-2]\getbuffer[L-3]\getbuffer[L-4]
  \stopchoice
  \getbuffer[answer]
\stopquestion

\startquestion
  Lorem ipsum dolor sit amet,
  \startchoice
    \getbuffer[L-3]\getbuffer[L-2]\getbuffer[L-3]\getbuffer[L-4]
  \stopchoice
  \getbuffer[answer]
\stopquestion

\startquestion
  Lorem ipsum dolor sit amet,
  \fastchoice{Lorem ipsum,{[*]Lorem ipsum},Lorem ipsum,Lorem ipsum}
  \getbuffer[answer]
\stopquestion

\startquestion
  Lorem ipsum dolor sit amet,
  \fastchoice{Lorem ipsum,{[*]Lorem ipsum},Lorem ipsum,Lorem ipsum}
  \getbuffer[answer]
\stopquestion

\startquestion
  Lorem ipsum dolor sit amet,
  \startinlinechoice
    \getbuffer[AL-1]\getbuffer[AL-2]\getbuffer[AL-3]\getbuffer[AL-4]
  \stopinlinechoice
  \getbuffer[answer]
\stopquestion

\startquestion
  Lorem ipsum dolor sit amet,
  \fastchoicealt{Lorem ipsum,{[*]Lorem ipsum},Lorem ipsum,Lorem ipsum}
  \getbuffer[answer]
\stopquestion

\startquestion
  Lorem ipsum dolor sit amet,
  \fastchoicealt{Lorem ipsum,{[*]Lorem ipsum},Lorem ipsum,Lorem ipsum}
  \getbuffer[answer]
\stopquestion

\section{Problem}

\startbuffer
\startpitem[answer={ccc,will be replace}]
   Lorem ipsum dolor sit amet,
   \fillin{Lorem ipsum,the real answer}
\stoppitem
\getbuffer[answer]
\stopbuffer

\setupquestion[answer=cxxx,showanswer=false,point=2,showpoint=false]

\startquestion
  Lorem ipsum dolor sit amet,
  \startproblem
    {\setupfillin[empty=number, n=10]\getbuffer}
    {\setupfillin[empty=yes,    n=10]\getbuffer}
    {\setupfillin[empty=none,   n=10]\getbuffer}
    {\setupfillin[empty=number, n=*]\getbuffer} 
    {\setupfillin[empty=yes,    n=*]\getbuffer}
    {\setupfillin[empty=none,   n=*]\getbuffer}
    {\setupfillin[empty=number, align=autoright,n=10]\getbuffer}
    {\setupfillin[empty=yes,    align=autoright,n=10]\getbuffer}
    {\setupfillin[empty=none,   align=autoright,n=10]\getbuffer}
    {\setupfillin[empty=number, align=autoright,n=*]\getbuffer}
    {\setupfillin[empty=yes,    align=autoright,n=*]\getbuffer}
    {\setupfillin[empty=none,   align=autoright,n=*]\getbuffer}
  \stopproblem
\stopquestion

\setupquestion[answer=cxxx,showanswer=true,point=2,showpoint=true]

\startquestion
  Lorem ipsum dolor sit amet,
  \startproblem
    \startpitem[answer={ccc,will be replace},point=5]
               Lorem ipsum dolor sit amet,
               \fillin{Lorem ipsum,the real answer 1}
               \fillin{Lorem ipsum,the real answer 2}\stoppitem
    {\setupfillin[empty=number, n=10]\getbuffer}
    {\setupfillin[empty=yes,    n=10]\getbuffer}
    {\setupfillin[empty=none,   n=10]\getbuffer}
    {\setupfillin[empty=number, n=*]\getbuffer} 
    {\setupfillin[empty=yes,    n=*]\getbuffer}
    {\setupfillin[empty=none,   n=*]\getbuffer}
    {\setupfillin[empty=number, align=autoright,n=10]\getbuffer}
    {\setupfillin[empty=yes,    align=autoright,n=10]\getbuffer}
    {\setupfillin[empty=none,   align=autoright,n=10]\getbuffer}
    {\setupfillin[empty=number, align=autoright,n=*]\getbuffer}
    {\setupfillin[empty=yes,    align=autoright,n=*]\getbuffer}
    {\setupfillin[empty=none,   align=autoright,n=*]\getbuffer}
  \stopproblem
\stopquestion

\section{Material}

\setupquestion[answer=]

\setupmaterial[title][color=green]
\setupmaterial[author][color=gray]
\setupmaterial[source][color=red]
\setupmaterial[number][numberconversion=I]
\setupmaterial[indicator][numberconversion=A]
\startmaterial[title={Knuth 1},author={Mos},source={Yelu}]
\input knuth\relax\indicator{Some Texts 1} \indicator{Some Texts 2} 
\stopmaterial
\startquestion
  \indicator{Some Texts 1}
  \fastchoice{Some Texts,Some Texts,Some Texts,Some Texts}
\stopquestion
\startquestion
  \indicator{Some Texts 2}
  \fastchoice{Some Texts,Some Texts,Some Texts,Some Texts}
\stopquestion

\startmaterial[title={Knuth 2},author={Mos},source={Yelu}]
\input knuth\relax\indicator{Some Texts 3} \indicator{Some Texts 4}
\stopmaterial
\startquestion
  \indicator{Some Texts 3}
  \fastchoice{Some Texts,Some Texts,Some Texts,Some Texts}
\stopquestion
\startquestion
  \indicator{Some Texts 4}
  \fastchoice{Some Texts,Some Texts,Some Texts,Some Texts}
\stopquestion

\section{Close}

\startclose
On Oct. 11, hundreds of runners competed in a cross-country race in Minnesota.
Melanie Bailey should have \closechoice[designed,followed,changed,finished] the
course earlier than she did. Her \closechoice[delay,chance,trouble,excuse] came
because she was carrying a \closechoice[judge,volunteer,classmate,competitor]
across the finish line.

As reported by a local newspaper, Bailey was more than two-thirds of the way
through her \closechoice[race,school,town,training] when a runner in front of
her began crying in pain. She \closechoice[agreed,returned,stopped,promised]
to help her fellow runner, Danielle Lenoue. Bailey took her arm to see if she
could walk forward with \closechoice[courage,aid,patience,advice] . She couldn't.
Bailey then \closechoice[went away,stood up,stepped aside,bent down] to let Lenoue
climb onto her back and carried her all the way to the finish line, then another
300 feet to where Lenoue could get \closechoice[medical,public,constant,equal] attention.

Once there, Lenoue was \closechoice[interrupted,assessed,identified,appreciated]
and later taken to a hospital, where she learned that she had serious injuries in
one of her knees. She would have struggled with extreme \closechoice[hunger,pain,cold,tiredness]
to make it to that aid checkpoint without Bailey's help.

As for Bailey, she is more \closechoice[worried,ashamed,confused,discouraged]
about why her act is considered a big \closechoice[game,problem,lesson,deal] .
"She was just crying. I couldn't \closechoice[leave,cure,bother,understand] her,"
Bailey told the reporter. "I feel like I was just doing the right thing.”

Although the two young women were strangers before the \closechoice[ride,test,meet,show] ,
they've since become friends. Neither won the race, but the \closechoice[secret,display,benefit,exchange]
of human kindness won the day.
\stopclose


\section{Score}

\startanswer
 some text \score{2}

 some text \score{4}
 
 \cdotfill\resetscore
 
 \setupscore[score=calculation]
 some text \score[type=cdotfill]{2}

 some text \score[type=space]{4}
 
 some text \score[type=hfill]{3}
 
 \cdotfill\resetscore
 
 \setupscore[score=complex]
 some text \score{2}

 some text \score{4}
 
 some text \score{3}
\stopanswer
\stoptext