\usemodule[memos]
   [typescript=adobe,
    latinfontname=texgyre,
    fontname=adobehans,
    fontsize=11pt,
    themecolor=green,
    layout=moderate,
    tocsstyle=line,
    chapterstyle=simple,
    hdrstyle=foemarginalt]
\usemodule[basicexam][mode=teacher]
\starttext

\placecontent

\chapter{環境設置}

\section{模塊設置}

\starttyping
\usemodule[basicexam][mode=student]
\stoptyping

可以通過修改 mode 的值來設定師生兩版。目前有 student、teacher 兩種屬性。

設定爲 student 之後,默認隱藏答案。設定爲 teacher 則反之。

\section{繪製卷頭}
\setuptype[lines=hyphenated]
\setuptyping[space=normal]

下面的命令簡單地展示了如何定義、設定並繪製一個新卷頭。
\startlists
\item 通過 \type{\definepapertitle[PaperTitleName]} 命令可以新建卷頭信息,
\item 通過  \type{\setuppapertitle[PaperTitleName]} 命令可以設定卷頭信息,
\item 通過   \type{\makepapertitle[PaperTitleName]} 命令可以繪製卷頭結果。
\stoplists

\startbuffer
\definepapertitle[papertitles]
\setuppapertitle [papertitles][
    n=5,              % 定义需要在试卷标题处需要显示多少元素,
    typi=secret,      % 同时,自动定义相应数量的元素命令
    typii=title,      % 使用typi typii typiii typiv ...
    typiii=subject,   % 定义每个元素的名称,同时自动生成相关样式化命令
    typiv=information,% X Xstyle Xalign beforeX afterX vspacetypi
    typv=notice,
    secretstyle=\ss,
    titlestyle=\ssa,
    subjectstyle=\ssb,
    informationstyle=\ttx,
    noticestyle=\rm\it,
    secretalign=flushleft,
    titlealign=center,
    subjectalign=center,
    informationalign=center,
    noticealign=flushleft,
    secret={绝密 ★ 启用前},
    title={2021 年普通高等学校招生全国统一考试},
    subject={日语},
    information={总分:150 分,考试时间:120 分钟},
    notice={注意事项:
      \startitemize[n,packed,joinedup]
        \item 答题前,务必将自己的姓名、准考证号填写在%
              答题卡规定的位置上。%
        \item 答选择題时,必须使用 2B 铅笔将答题卡上对%
              应题目的答案标号涂黑。如需改动,用橡皮擦%
              擦干净后,再选涂其它答案标号。答非选择题%
              时,必须使用 0.5 毫米黑色签字笔,将答案%
              书写在答题卡规定的位置上。所有題目必须在答%
              题卡上作答,在试题卷上答题无效。 %
        \item 考试结束后,将试题卷和答题卡一并交回。
        \stopitemize},
    ]
\stopbuffer

\typebuffer

\definepapertitle[papertitles]
\setuppapertitle [papertitles][
    n=5,              % 定义需要在试卷标题处需要显示多少元素,
    typi=secret,      % 同时,自动定义相应数量的元素命令
    typii=title,      % 使用typi typii typiii typiv ...
    typiii=subject,   % 定义每个元素的名称,同时自动生成相关样式化命令
    typiv=information,% X Xstyle Xalign beforeX afterX vspacetypi
    typv=notice,
    secretstyle=\ss,
    titlestyle=\ssa,
    subjectstyle=\ssb,
    informationstyle=\ttx,
    noticestyle=\rm\it,
    secretalign=flushleft,
    titlealign=center,
    subjectalign=center,
    informationalign=center,
    noticealign=flushleft,
    secret={绝密 ★ 启用前},
    title={2021 年普通高等学校招生全国统一考试},
    subject={日语},
    information={总分:150 分,考试时间:120 分钟},
    notice={注意事项:
      \startitemize[n,packed,joinedup]
        \item 答题前,务必将自己的姓名、准考证号填写在%
              答题卡规定的位置上。%
        \item 答选择題时,必须使用 2B 铅笔将答题卡上对%
              应题目的答案标号涂黑。如需改动,用橡皮擦%
              擦干净后,再选涂其它答案标号。答非选择题%
              时,必须使用 0.5 毫米黑色签字笔,将答案%
              书写在答题卡规定的位置上。所有題目必须在答%
              题卡上作答,在试题卷上答题无效。 %
        \item 考试结束后,将试题卷和答题卡一并交回。
        \stopitemize},
    ]
\framed[width=\textwidth,align=flushleft]{\makepapertitle[papertitles]}

\section{标题}

默认设定了四级标题。不同于使用 \type{\chapter}、\type{\section} 的传统命令,
为了创建可分离的目录,同时也为了区分普通文档和试卷文档,我们使用 
\type{\tu}、\type{\ego}、\type{\isea}、\type{\heu} 来标记各级标题。

默认情况下,设定 \type{\tu} 用于排版试卷标题,
\type{\ego}、\type{\isea}、\type{\heu} 用来排版试卷内部各级标题。

想要生成目录,可以使用 \type{\placetocexam}。默认只生成试卷标题,内部各级标
题并不包含在内。想要调整目录样式,使用 \CONTEXT 的目录设置命令即可。

\startbuffer \placetocexam \stopbuffer%%% TODO
%
\typebuffer

% \framed[width=\textwidth,align=flushleft]{\placetocexam}

\section{題目設置}

\def\cmd#1{{\tt #1}}
\def\examplewords{該環境位於 \currentitemgroup 。}

\subsection{question 環境命令}

\cmd{question} 環境命令可以用來設置題幹。
只需將題幹包裹在 \cmd{startquestion}  \cmd{stopquestion} 命令之間即可。

\startbuffer[eg_question]
\startquestion
\examplewords
\stopquestion
\startquestion
\examplewords
\stopquestion
\stopbuffer

\typebuffer[eg_question]

\framed[width=\textwidth,align={flushleft,lohi}]{
\getbuffer[eg_question]
}

\cmd{question} 環境命令繼承了大部分 \cmd{itemgroup} 環境的選項設置。
因此,可以像修改 \cmd{itemgroup} 一樣,使用 \cmd{setupquestion} 來修改該環境。

\bgroup
\startbuffer
\setupquestion[color=green,style=\ss,start=22]
\stopbuffer

\typebuffer
\getbuffer

\framed[width=\textwidth,align={flushleft,lohi}]{
	\getbuffer[eg_question]}
\egroup

同時,該環境命令還設置了額外的選項設置,選項功能和名稱相同:

\startpoints[horizontal,three]
\startitem showpoint   \stopitem \startitem point       \stopitem
\startitem pointstyle  \stopitem \startitem pointcolor  \stopitem
\startitem pointlabel  \stopitem \startitem             \stopitem
\startitem showanswer  \stopitem \startitem answer      \stopitem
\startitem answerstyle \stopitem \startitem answercolor \stopitem
\stoppoints

\startbuffer
\startquestion
[showpoint=true,point=5,pointlabel={points},
 showanswer=true,answer={new answer},answerstyle={\tt},]
\examplewords
\stopquestion
\stopbuffer

\typebuffer

\framed[width=\textwidth,align={flushleft,lohi}]{\getbuffer}

\subsection{problem 環境命令}

\cmd{problem} 環境命令可以設置多個聚合問題,該環境具有一個特定的子命令 \cmd{pitem} 來列明每個問題。可以用來設置填空、問答等題目。

\startbuffer
\startproblem
\startpitem \examplewords \stoppitem
\startpitem \examplewords \stoppitem
\startpitem \examplewords \stoppitem
\stopproblem
\stopbuffer

\typebuffer

\framed[width=\textwidth,align={flushleft,lohi}]{\getbuffer}

正如 \cmd{question} 環境命令繼承了大部分 \cmd{itemgroup} 環境的選項設置。
因此,可以像修改 \cmd{setupquestion} 一樣,使用 \cmd{setupproblem} 來修改該環境。但是不同的是,默認情況下,並沒有爲 \cmd{problem} 環境命令設置特別的選項設置。這是因爲在設計該命令時,默認認爲該環境具有一系列的子問題。由於每個問題都會有自己的答案,因此,\cmd{problem} 並不具有特別的選項設置。

出於原本的設計目的,每個問題的答案和分數等選項設計都可以通過 \cmd{pitem} 來設定。

\startbuffer
\startproblem
\startpitem[showpoint=true,point=5]         \examplewords \stoppitem
\startpitem[showanswer=true,answer={newer}] \examplewords \stoppitem
\startpitem[] \examplewords \stoppitem
\stopproblem
\stopbuffer

\typebuffer

\framed[width=\textwidth,align={flushleft,lohi}]{\getbuffer}

不同於 \cmd{question} 環境所有的序號都是連續的,\cmd{problem} 環境每次開始後都是重新進行計數。

如果想要爲 \cmd{problem} 環境添加題幹,可以將 \cmd{problem} 環境放置在 \cmd{question} 環境之中。

\startbuffer
\startquestion
  \examplewords
    \startproblem[left={(},right={)},distance=1em,stopper={}]
    \startpitem \examplewords \stoppitem
    \startpitem \examplewords \stoppitem
    \startpitem \examplewords \stoppitem
    \stopproblem
\stopquestion
\stopbuffer

\typebuffer

\framed[width=\textwidth,align={flushleft,lohi}]{\getbuffer}

如果想要設置填空題,可以使用 \cmd{fillin} 命令。該命令結合了 \cmd{textnote} 命令和 \cmd{underbar} 命令。主要具有這些選項設置:

\startpoints[horizontal,three]
\startitem type     \stopitem \startitem n      \stopitem
\startitem continue \stopitem \startitem empty  \stopitem
\startitem unit     \stopitem \startitem dy     \stopitem
\startitem method   \stopitem \startitem max    \stopitem
\startitem offset   \stopitem \startitem repeat \stopitem
\startitem left     \stopitem \startitem right  \stopitem
\startitem color    \stopitem \startitem width  \stopitem
\startitem order    \stopitem \startitem mp     \stopitem 
\startitem foregroundcolor \stopitem 
\startitem foregroundstyle \stopitem
\startitem rulethickness   \stopitem 
\stoppoints

上述鍵值中,需要進一步解釋的鍵值主要有:

\startpoints
\item \type{type}  默認設置了: 
                   \type{underbar},
                   \type{textnote},
                   \type{void} 三種樣式;
\item \type{empty} 默認設置了:
                   \type{yes},
                   \type{no},
                   \type{number} 三種樣式;
\item \type{mp}    的值可以使用系統默認的:
                   \type{rules:under:dots},
               	   \type{rules:under:random},
                   \type{rules:under:dash},
                   \type{rules:under:wave} 幾種樣式
\stoppoints

需要注意的是,\type{textnote} 雖然在外觀上沒有特別的違和感,但目前來看,textnote並不會把答案放置在正確的位置,/fillin 命令目前無法完全兼容。

除了上述的鍵值,還提供了一個可以獲取統一環境下答案的命令:\cmd{getanswerfromfillin}。該命令會獲取臨近 \cmd{fillin} 的內容,並輸出爲答案。

\startbuffer
\setupanswer[showanswer=true]
\startquestion
  \examplewords 
    \setuppitem[answer=\getanswerfromfillin]
    \startproblem[left={(},right={)},distance=1em,stopper={}]
    \startpitem 該環境位於 \fillin{env_question:1}。 \stoppitem
    \startanswer[answer=\getanswerfromfillin]
    	\input knuthmath
    \stopanswer
    \startpitem 該環境位於 \fillin[mp=rules:under:wave]{env_question:1}。 \stoppitem
    \startanswer[answer=\getanswerfromfillin]
    	\input knuthmath
    \stopanswer
    \startpitem 
    	該環境位於 \fillin[type=textnote,empty=number]{env_question:1}。 注意,textnote 並不會把答案放置在正確的位置,\cmd{fillin} 命令目前無法完全兼容。
    \stoppitem
    \startanswer[answer=\getanswerfromfillin]
    	\input knuthmath
    \stopanswer
    \startpitem 該環境位於 \fillin[type=void]{}。 \stoppitem
    \startanswer[answer=\getanswerfromfillin]
    	\input knuthmath
    \stopanswer
    \startpitem 該環境位於 \fillin[type=void,empty=yes,left={(},right={)}]{}。 \stoppitem
    \startanswer[answer=\getanswerfromfillin]
    	\input knuthmath
    \stopanswer
    \stopproblem
\stopquestion
\stopbuffer

\typebuffer

\framed[width=\textwidth,align={flushleft,lohi}]{\getbuffer}

\subsection{material 環境命令}

顧名思義,\cmd{material} 環境用來放置大段文本材料。該環境命令因爲特殊性未繼承任何命令設置選項(\cmd{setupmaterial \[number\]} 標題序號命令繼承了 \cmd{setupcounter} 的設置選項,用來快速調整標題序號),所有選項設置都是特製的。

主要的特殊設置如下:

\startpoints
\item \cmd{setupmaterial} 可以設置整體環境,具體擁有如下鍵值:
	\startpoints[horizontal,three]
      \startitem align       \stopitem
      \startitem style       \stopitem
      \startitem color       \stopitem
      \startitem spacebefore \stopitem
      \startitem spaceafter  \stopitem
      \startitem title       \stopitem
	  \startitem author      \stopitem
      \startitem source      \stopitem
      \startitem indicator   \stopitem
	\stoppoints
\item \cmd{setupmaterial \[number\]} 可以用來設置標題序號,
\item \cmd{setupmaterial \[title\]}  可以用來設置標題樣式。
\item \cmd{setupmaterial \[author\]} 可以用來設置作者樣式,
\item \cmd{setupmaterial \[source\]} 可以用來設置文章來源樣式。
\stoppoints

\startbuffer
\setupmaterial[title][color=green]
\startmaterial[title={Knuth},author={Mos},source={Yelu}]
\input knuth
\stopmaterial
\stopbuffer

\typebuffer

\framed[width=\textwidth,align={flushleft,lohi}]{\getbuffer}

\subsection{choice   環境命令}

該環境命令只能置於 \cmd{question} 環境之下。該環境的子命令 \cmd{citem} 目前必須使用 

\starttyping
\startcitem {選項內容} \stopcitem
\stoptyping

的形式,如此纔可以正確的排版選項\footnote{該命令是通過獲取選項內容長度來計算選項的排版方式。}。

如果想要設置多選題,只需要對每個正確答案添加 \type{[*]} 即可。

\startbuffer
\setupquestion[showanswer=true]
\startquestion[answer=\getanswerfromchoice]
    \examplewords
    \startchoice
        \startcitem[*]{Some Words}\stopcitem 
        \startcitem{Some Words}\stopcitem
        \startcitem[*]{Some Words}\stopcitem 
        \startcitem{Some Words}\stopcitem
    \stopchoice
\stopquestion
\startquestion[answer=\getanswerfromchoice]
    \examplewords
    \startchoice
        \startcitem{Some More Words}\stopcitem 
        \startcitem{Some More Words}\stopcitem
        \startcitem[*]{Some More Words}\stopcitem 
        \startcitem[*]{Some More Words}\stopcitem
    \stopchoice
\stopquestion
\startquestion[answer=\getanswerfromchoice]
    \examplewords
    \startchoice
        \startcitem{Some More More Words}\stopcitem 
        \startcitem[*]{Some More More Words}\stopcitem
        \startcitem{Some More More Words}\stopcitem 
        \startcitem[*]{Some More More Words}\stopcitem
    \stopchoice
\stopquestion
\stopbuffer

\typebuffer

\framed[width=\textwidth,align={flushleft,lohi}]{\getbuffer}

特別的,設定了一個脆弱的選項命令。該命令可以不通過界定花括號來排版選項,但可能會引發未知問題,特別是選項中含有空格時。

\startbuffer
\setupquestion[showanswer=true]
\startquestion[answer=\getanswerfromchoice]
    \examplewords
    \startchoice
        \fragilecitem[*]{SomeWords}
        \fragilecitem SomeWords
        \fragilecitem[*]{SomeWords}
        \fragilecitem SomeWords
    \stopchoice
\stopquestion
\stopbuffer

\typebuffer

\framed[width=\textwidth,align={flushleft,lohi}]{\getbuffer}

\subsection{close   環境命令}

用於生成完形填空。

\startclose
On Oct. 11, hundreds of runners competed in a cross-country race in Minnesota. 
Melanie Bailey should have \closechoice[designed,followed,changed,finished] the 
course earlier than she did. Her \closechoice[delay,chance,trouble,excuse] came 
because she was carrying a \closechoice[judge,volunteer,classmate,competitor] 
across the finish line.

As reported by a local newspaper, Bailey was more than two-thirds of the way 
through her \closechoice[race,school,town,training] when a runner in front of 
her began crying in pain. She \closechoice[agreed,returned,stopped,promised] 
to help her fellow runner, Danielle Lenoue. Bailey took her arm to see if she 
could walk forward with \closechoice[courage,aid,patience,advice] . She couldn't. 
Bailey then \closechoice[went away,stood up,stepped aside,bent down] to let Lenoue 
climb onto her back and carried her all the way to the finish line, then another 
300 feet to where Lenoue could get \closechoice[medical,public,constant,equal] attention.

Once there, Lenoue was \closechoice[interrupted,assessed,identified,appreciated] 
and later taken to a hospital, where she learned that she had serious injuries in 
one of her knees. She would have struggled with extreme \closechoice[hunger,pain,cold,tiredness] 
to make it to that aid checkpoint without Bailey's help.

As for Bailey, she is more \closechoice[worried,ashamed,confused,discouraged] 
about why her act is considered a big \closechoice[game,problem,lesson,deal] . 
"She was just crying. I couldn't \closechoice[leave,cure,bother,understand] her,"
Bailey told the reporter. "I feel like I was just doing the right thing.”

Although the two young women were strangers before the \closechoice[ride,test,meet,show] , 
they've since become friends. Neither won the race, but the \closechoice[secret,display,benefit,exchange] 
of human kindness won the day.
\stopclose

\subsection{answer   環境命令}

\cmd{answer} 是爲師生兩版設置的命令。但目前該命令和相關設置還具有諸多不足之處。只需要將該環境置於題目下方,即可獲取當前題目設定的答案,並可以爲其編寫答案解析。

\startbuffer
\startquestion[showanswer=true,answer={answer for \currentitemgroup }]
    \examplewords 注意:\cmd{answer} 可以獲取父環境或同級環境的答案而不必再次設置。
    \startanswer[point=12]
    \input knuthmath
    \stopanswer
\stopquestion
\startquestion[showanswer=true,answer=\getanswerfromchoice]
    \examplewords 注意:\cmd{answer} 可以獲取父環境或同級環境的答案而不必再次設置。
    \startchoice
        \startcitem{Some Words}\stopcitem 
        \startcitem{Some Words}\stopcitem
        \startcitem[*]{Some Words}\stopcitem 
        \startcitem{Some Words}\stopcitem
    \stopchoice
    \startanswer[point=1]
    \input knuthmath
    \stopanswer
\stopquestion
\stopbuffer

\typebuffer

\framed[width=\textwidth,align={flushleft,lohi}]{\getbuffer}

\startbuffer
\startquestion
  \examplewords
    \startproblem[showanswer=true,left={(},right={)},distance=1em,stopper={}]
    \startpitem[showanswer=true,answer={Pitem 1}] \examplewords \stoppitem
    \startanswer[point=10]
    \input knuthmath
    \stopanswer
    \startpitem[showanswer=true,answer={Pitem 2}] \examplewords \stoppitem
    \startanswer[point=20]
    \input knuthmath
    \stopanswer
    \startpitem[showanswer=true,answer={Pitem 3}] \examplewords \stoppitem
    \startanswer[point=30]
    \input knuthmath
    \stopanswer
    \stopproblem
\stopquestion
\stopbuffer

\typebuffer

\framed[width=\textwidth,align={flushleft,lohi}]{\getbuffer}

該環境具有一系列的相關選項設置:

\startpoints[horizontal,three]
    \startitem showanswer  \stopitem
    \startitem answer      \stopitem
    \startitem answerstyle \stopitem
    \startitem answercolor \stopitem
    \startitem showpoint   \stopitem
    \startitem point       \stopitem
    \startitem pointstyle  \stopitem
    \startitem pointcolor  \stopitem
    \startitem label       \stopitem
    \startitem labelstyle  \stopitem
    \startitem before      \stopitem
    \startitem after       \stopitem
    \startitem afteranswer \stopitem
\stoppoints

\subsection{繪製作文格}

\makewritingbox

\chapter{示例}

\startbuffer
\startquestion
    \examplewords
  \startchoice
     \startcitem    \examplewords \stopcitem
     \startcitem    \examplewords \stopcitem
     \startcitem[*] \examplewords \stopcitem
     \startcitem    \examplewords \stopcitem
  \stopchoice
  \startanswer[point=2]
    \examplewords
  \stopanswer
\stopquestion
\stopbuffer

\typebuffer

\getbuffer

\blackrule[color=green,width=1tw,height=.5pt]

\startbuffer
\startquestion
    \examplewords
  \startproblem[left={(},right={)},distance=1em,stopper={}]
     \startpitem[answer=p1::\examplewords] \examplewords \stoppitem
       \startanswer[point=1]\examplewords\stopanswer
     \startpitem[answer=p2::\examplewords] \examplewords \stoppitem
       \startanswer[point=2]\examplewords\stopanswer
     \startpitem[answer=p3::\examplewords] \examplewords \stoppitem
       \startanswer[point=3]\examplewords\stopanswer
     \startpitem[answer=p4::\examplewords] \examplewords \stoppitem
       \startanswer[point=4]\examplewords\stopanswer
  \stopproblem
\stopquestion
\stopbuffer

\typebuffer

\getbuffer

\blackrule[color=green,width=1tw,height=.5pt]

\def\fillinexamplewords{該環境位於 \fillin{f::\currentuserorder}。}

\startbuffer
\startquestion
    \examplewords
    \setuppitem[answer=\getanswerfromfillin]
  \startproblem[left={(},right={)},distance=1em,stopper={}]
     \startpitem \fillinexamplewords \stoppitem
       \startanswer[point=1]\examplewords\stopanswer
     \startpitem \fillinexamplewords \stoppitem
       \startanswer[point=2]\examplewords\stopanswer
     \startpitem \fillinexamplewords \stoppitem
       \startanswer[point=3]\examplewords\stopanswer
     \startpitem \fillinexamplewords \stoppitem
       \startanswer[point=4]\examplewords\stopanswer
  \stopproblem
\stopquestion
\stopbuffer

\typebuffer

\getbuffer

\blackrule[color=green,width=1tw,height=.5pt]

\startbuffer
\setupmaterial[number][way=bychapter]
\startmaterial[title=title,author=author,source=source]
  \input knuthmath \indicator{GGGG}
  \input knuthmath \indicator{HHHH}
  \input knuthmath \indicators
\stopmaterial
\startquestion
    \examplewords \indicator{GGGG}
  \startchoice
     \startcitem    \examplewords \stopcitem
     \startcitem    \examplewords \stopcitem
     \startcitem[*] \examplewords \stopcitem
     \startcitem    \examplewords \stopcitem
  \stopchoice
  \startanswer[point=2]
    \examplewords
  \stopanswer
\stopquestion
\startquestion
    \examplewords \indicator{HHHH}
  \startchoice
     \startcitem    \examplewords \stopcitem
     \startcitem[*] \examplewords \stopcitem
     \startcitem    \examplewords \stopcitem
     \startcitem    \examplewords \stopcitem
  \stopchoice
  \startanswer[point=3]
    \examplewords
  \stopanswer
\stopquestion
\stopbuffer

\typebuffer

\getbuffer

\blackrule[color=green,width=1tw,height=.5pt]

\startbuffer
\startclose
  \input knuthmath \closechoice[AAA,BBB,CCC,DDD]
  \input knuthmath \closechoice[EEE,BBB,FFF,DDD]
\stopclose
\stopbuffer

\typebuffer

\getbuffer

\blackrule[color=green,width=1tw,height=.5pt]

\startbuffer
\startoptclose
  \input knuthmath \ocitem{AAA,BBB,CCC,DDD}
  \input knuthmath \ocitem{EEE,BBB,FFF,DDD}
  \input knuthmath \ocitem{EEE,GGG,FFF,HHH}
\stopoptclose
\stopbuffer

\typebuffer

\getbuffer

\blackrule[color=green,width=1tw,height=.5pt]

\startbuffer
\startoptclose
  \input knuthmath \specialocitem[answer={AAA,BBB,CCC,DDD}]
  \input knuthmath \specialocitem[answer={EEE,BBB,FFF,DDD}]
  \input knuthmath \specialocitem[answer={EEE,GGG,FFF,HHH}]
\stopoptclose
\stopbuffer

\typebuffer

\getbuffer

\chapter{TODO List}

%\startTODO
  \startlists
     \item 連線題
     \item 是否應該爲模塊添加字體設置選項?
     \item 佈局設置(雙開)
     \item 雙語設置
     \item 自動計算分值?
  \stoplists
%\stopTODO
\stoptext